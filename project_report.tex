\documentclass{article}

%% Page Margins %%
\usepackage{geometry}
\geometry{
    top = 0.75in,
    bottom = 0.75in,
    right = 0.75in,
    left = 0.75in,
}

\usepackage{amsmath}
\usepackage{graphicx}
\usepackage{parskip}

\title{Assembly Project: Tetris}

% TODO: Enter your name
\author{Yehyun Lee & Hisham}

\begin{document}
\maketitle

\section{Instruction and Summary}

\begin{enumerate}

    \item Which milestones were implemented? \\
    % TODO: List the milestone(s) and in the case of 
    %       Milestones 4 & 5, list what features you 
    %       implemented, sorted into easy and hard 
    %       categories.
    - Milestones 1, 2, 3 has been completed!

    For the snapshot of progress take a look at our Github page. Please contact yehyun.lee@mail.utoronto.ca for access to private repo.
    https://github.com/YehyunLee/AssemblyTetris

    \item How to view the game:
    % TODO: specify the pixes/unit, width and height of 
    %       your game, etc.  NOTE: list these details in
    %       the header of your breakout.asm file too!
    
    \begin{enumerate}

    \item Run the tetris.asm file in a MIPS IDE such as Saturn or Mars


    \end{enumerate}

    

\begin{figure}[ht!]
    \centering
    % \includegraphics[width=0.3\textwidth]{name.png}
    \caption{caption}
    \label{Instructions}
\end{figure}

\item Game Summary:
% TODO: Tell us a little about your game.
\begin{itemize}
\item Pressing `W" would rotate the tetrominoes 90 degrees clockwise
\item Pressing `S" would decrease the position of incoming tetrominoes by 1 until it collides
\item If it collides, a new tetromino is loaded into the screen.
\item For this game version, the program is design to go on until program stops 
\item Program can be stop by pausing the run or clicking "Q" on the keyboard
\item For now, only a specific tetormino, specified will get dropped. However, in the next update, we will implement the random function to randomize tetrominoes being created.
\end{itemize}

\begin{verbatim}
    # Major variables:
    # lw $s0 for paint (sw)
    # li $s1 for paint counter (need this for general use)
    # li $s2 for what TETRO, ex) O, J, T, using int; refer to image.
    # li $s3 for what ANGLE ex) 0 is default, 1 is one 90 roration upto 3.
    # li $s4 OTetrominoX
    # li $s5 OTetrominoY
    # lw $s6, ADDR_DSPL
    # lw $s7, ADDR_KBRD
    # a3 for collision
\end{verbatim}


    
\end{enumerate}

\section{Attribution Table}
% TODO: If you worked in partners, tell us who was 
%       responsible for which features. Some reweighting 
%       might be possible in cases where one group member
%       deserves extra credit for the work they put in.



\begin{center}
\begin{tabular}{|| c | c ||}
\hline
 Yehyun Lee (1008992217) &  Aung Zwe Maw (1008604949) \\ 
 \hline
 [MEDIUM] Coded the background: grid and 3 walls & [HARD] Implemented Original tetromino drawings\\
 \hline
 [HARD] Designed and Coded Collision Logic & [HARD] Also created every possible rotated tetromino drawing\\
 \hline
 [HARD] Movement W, A, S, D & ←Hisham helped me debugging with CodeTogether until 2AM.\\ 
 \hline
 [EASY] Coded Keyboard Input & [MEDIUM] Rotation is consisten at a consistent point of origin\\ 
 \hline
 [HARD] General Game Loop Logic Flow:\\Saving Tetrominoes Information,\\
 Loading and Handles All Game States & [HARD] Rotation code so screen reloads with rotated Tetromino\\
 \hline
\end{tabular}

\begin{figure}[ht!]
    \centering
    \includegraphics[width=0.5\textwidth]{tetris_diagram.png}
    \caption{Draft Diagram by Yehyun}
    \label{f:1}
\end{figure}

\begin{figure}[ht!]
    \centering
    \includegraphics[width=0.5\textwidth]{diagram2.png}
    \caption{Draft Diagram by Yehyun: Outdated Game Logic}
    \label{f:1}
\end{figure}


\begin{figure}[ht!]
    \centering
    \includegraphics[width=0.5\textwidth]{diagram3.png}
    \caption{Snapshot of Progress}
    \label{f:1}
\end{figure}


\end{center}


% TODO: Fill out the remainder of the document as you see 
%       fit, including as much detail as you think 
%       necessary to better understand your code. 
%       You can add extra sections and subsections to 
%       help us understand why you deserve marks for 
%       features that were more challenging than they
%       might initially seem.


\end{document}